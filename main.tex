%Base Content generated with the help of AI.

\documentclass{beamer}
\usepackage{listings}
% Presentation theme
\usetheme{default}
\usecolortheme{seagull}
% Listings settings for code display
\lstset{
    basicstyle=\ttfamily,
    columns=fullflexible,
    frame=single,
    breaklines=true
}
% Title slide
\title{Pros and Cons of Using LaTeX}
\author{Your Name}
\date{\today}


\begin{document}

\begin{frame}
    \titlepage
\end{frame}

% Pros slides
\begin{frame}{Beautiful Professional Typesetting}
    I only had to type this text and LaTeX rendered it beautifully on this slide. I didn't even have to choose the font. 
    \begin{example}
    I can also choose from built-in modules and templates, like this "example" environment. 
    \end{example}
\end{frame}

\begin{frame}{Equations in LaTeX}
Equations can be embedded in-line
\begin{example}
    The famous equation: $E=mc^2$
\end{example}
or
\begin{example}
    They can be given their own space on the page (and auto-numbered)
    \begin{equation}
        F = G \cdot \frac{m_1 \cdot m_2}{r^2}
    \end{equation}
\end{example}
\end{frame}

\begin{frame}[fragile]{Embed code snippets}\label{slide:code}
    Use the \lstinline|listings| package to embed code snippets.
    \begin{example}
      \begin{lstlisting}[mathescape]
    \begin{equation}
        F = G \cdot \frac{m_1 \cdot m_2}{r^2}
    \end{equation}
    \end{lstlisting} will render this popular equation \cite{einstein1905}
    \begin{equation}\label{eq:example}
        F = G \cdot \frac{m_1 \cdot m_2}{r^2}
    \end{equation}
    
    \end{example}
    
    
\end{frame}
\begin{frame}{Cross-Referencing}
    LaTeX enables accurate cross-referencing of figures, tables, equations, and sections, maintaining consistency and reducing errors.
    
    \begin{example}
        LaTeX allows easy cross-referencing to equations, such as Equation~\ref{eq:example} from the slide \ref{slide:code}.
    \end{example}
\end{frame}

\begin{frame}{Version Control}
   Plain text-files are ideal for Git. Can also use overleaf and other tools.
    
    \begin{example}
        Collaborators can use Git to track changes in a LaTeX document over time.
    \end{example}
\end{frame}

\begin{frame}{Customization}
    LaTeX offers extensive customization options for formatting and styling, allowing you to create unique and professional-looking documents.
    
    \begin{example}
        Customize the document layout, fonts, colors, and more to create a distinctive style.
    \end{example}
\end{frame}

\begin{frame}{Community and Templates}
    LaTeX has a vast community, providing resources, templates, and packages that simplify tasks and enhance document appearance.
    
    \begin{example}
        Utilize LaTeX templates for presentations, research papers, CVs, and more, shared by the community.
    \end{example}
\end{frame}

\begin{frame}{References}\label{slide:references}
Finally, it's super-simple to embed and format citations \cite{latexcompanion}
\begin{example}
    \bibliographystyle{ieeetr}
    \bibliography{example_bib}
\end{example}
\end{frame}

% Cons slide
\begin{frame}{Cons}
    \begin{itemize}
        \item Learning Curve
        \item Real-Time Collaboration
        \item WYSIWYG Limitations
        \item Limited Graphics Support
        \item Complex Layouts
        \item Customization Complexity
        \item Document Maintenance
    \end{itemize}
\end{frame}

\end{document}
